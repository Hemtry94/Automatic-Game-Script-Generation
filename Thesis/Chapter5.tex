This chapter shows the results of the different techniques of level generation and rule generation. The resulted levels and games of the system are published on a website \footnote{http://www.amidos-games.com/puzzlescript-pcg/} to collect human feedback. It shows also a comparison between the system results with collected human statistics. In the following sections we start describing the input for level and rule generation followed by description of our experiments. By the end  compare its results with collected human statistics.

\section{Automated Player}
\secref{automatedPlayer} introduces new metric to solve the problem of having Player as one of the winning objects. Several experiments are done to detect the best possible weights for the three metrics and comparing the results with Lim et al.\cite{puzzleScriptGeneration} previous metrics.\\\par

The new player give different weights to each one of the three metrics. The distance between winning objects metric is the most important metric, other metrics contributes with less score (50\% decrease). The distance between player and winning objects metric contributes with a lower percentage than the new metric (50\% lower). This decrease is due to the fact that one of the winning objects is a rule object. The new metric measures the distance between the rule objects. This means this object contributes in both metrics at same time.\\\par

The following section will describe the different games and levels used to test both metrics followed by different experiments to measure the difference between both metrics.

\subsection{Input Description}\seclbl{gameDescription}
Forty handcrafted levels from five different games were used to test our the introduction of the new metric. The five games where completely different to ensure covering most of behaviors. Levels are also designed with different sizes and different ideas to cover different design aspect.\\\par

The five games are Sokoban, LavaGame, BlockFaker, GemGame, and DestroyGame.
\begin{itemize} \itemsep0pt \parskip0pt \parsep0pt
	\item \textbf{Sokoban:} The goal of the game is to place every single crate over a certain position. Player is supposed to push crates to achieve that goal.
	\item \textbf{LavaGame:} The goal of the game is to reach the exit. Most of the time the path towards the exit is blocked by lava. Player is supposed to move crates over lava to clear his way.
	\item \textbf{BlockFaker:} The goal of the game is to reach the exit. The path towards the exit is blocked by lots of crates. Player should push the crates to align them vertically or horizontally. Every three aligned crates are removed clearing the path towards the exit.
	\item \textbf{GemGame:} The goal of the game is to place at least one gem over one of several locations. Player can create gems by pushing crates. Every three aligned crates are replaced with a single diamond instead of the middle crate.
	\item \textbf{DestroyGame:} The goal of the game is to clear every single gem. Gems can be destroyed when they are aligned with two other crates vertically or horizontally. Player should push crates to reach that goal.
\end{itemize}

Each of the previous games have different rules, goals, and objects.
\subsection{Comparing Different Players}
Lim et al.\cite{puzzleScriptGeneration} automated player plays all the forty levels and it is compared with the introduced automated player in this work. \figref{automatedPlayerPerformance} shows the average number of states each player explored in each game before reaching goal.

\gfigure{Comparison between the number of explored states for different automated players}{automatedPlayerPerformance}{width=6.0in}{Images/automatedPlayerPerformance}

The new player outperforms original player in Sokoban game and Gem game but its almost the similar in the rest of the games. In games where the player is one of the winning objects, the new player just performed slightly better in BlockFaker, while LavaGame is slightly worse. The same happens with DestroyGame. The main reason behind that is the presence of Destroy behavior as the core of the game. For example, \figref{automatedPlayerProblem} have two crates and two lava, the metric measures the average distance between each crate and all lava. If the lower lava is destroyed first that remain us with one box with one lava which makes the metric a little bit worse than before (only long distance remains). The system will explore more states that will not lead to destroy of the lower lava.

\gfigure{LavaGame level cause the new player to do worse}{automatedPlayerProblem}{width=2.5in}{Images/automatedPlayerProblem}

Another drawback of the new metric is when levels have lots of unused objects. The new metric will make the player explores these objects which increase the number of explored states before reaching the goal object. This drawback is not so important as the main goal is to generate levels not solve them. The presence of unused objects may cause increase of the difficulty of generated levels.\\\par

The average solution length of each game is presented in \figref{automatedPlayerLength}. The new player produces slightly an overall shorter solutions than the original player. Comparing both \figref{automatedPlayerPerformance} and \figref{automatedPlayerLength} there is a correlation between the average number of explored states and the average solution length except for Sokoban. Sokoban doesn't follow same pattern due to being abstract with very small amount of objects and just one rule. This abstraction is the main reason both players can reach the goal in almost the same amount of steps.

\gfigure{Comparison between the average solution length for different automated players}{automatedPlayerLength}{width=6.0in}{Images/automatedPlayerLength}

\tabref{playerCorrelation} shows the correlation value between the average solution length and the average explored states by the two different players and the correlation is calculated at the bottom of the table. The values in the table is the difference between the original player and the new player. Positive values indicates the new player is better than the original and vice versa.

\begin{table}[!ht]
	\centering
	\begin{tabular}{|c|c|}
		\hline
		\textbf{The average solution length} & \textbf{The average explored states}\\
		\hline
		-0.875 & 11127.625\\
		\hline
		-1 & -3914\\
		\hline
		5 & 2736.625\\
		\hline
		12.875 & 16763.625\\
		\hline
		-4.75 & -6696.75\\
		\hline
		\textbf{Correlation} & 0.77127\\
		\hline
	\end{tabular}
	\caption{Correlation between the average explored states and the average solution length}
	\label{Table:playerCorrelation}
\end{table}

\section{Level Generation}
This section shows the results of different techniques introduced in \chref{Chapter4} for generating different levels. Automated player is used with fixed number of iteration. Automated player takes very long time to play the level. The automated player is limited to 5000 explored states to ensure fast execution and level quality.\\\par

The following section will discuss the required input data for the system to work, followed by the results for each of different approach used.

\subsection{Input Description}
From \chref{Chapter4}, level generation needs a game description and some level layouts. Five different games were tested with eight different level layouts.

\subsubsection{Game Descriptions}
The five different games are the same games described in \secref{gameDescription}. Sticking with same games as they covers different behaviors and winning rules.
\begin{itemize}
	\item \textbf{Sokoban:} The game have four main objects (Player, Crate, Target, and Wall). Crate and Target are the winning objects with Crate having a Move behavior. Wall is a solid object. The game has an All winning rule.
	
	\item \textbf{LavaGame:} The game has five main objects (Player, Lava, Crate, Target, and Wall). Player and Target are the winning objects. Lava and Crate are normal rule objects with Destroy behavior. Crate has a Move behavior too. Wall is solid object. The game has an All winning rule.
	
	\item \textbf{BlockFaker:} The game has five main objects (Player, Crate, Stopper, Target, and Wall). Player and Target are the winning objects. Stopper and Crate are normal rule objects with Crate having Move and Destroy behaviors. Wall is solid object. The game has an All winning rule.
	
	\item \textbf{GemGame:} The game has five main objects (Player, Crate, Gem, Target, and Wall). Gem and Target are the winning objects with Gem having Create behavior. Crate is a critical rule object with Move and Destroy behaviors. Wall is a solid Object. The game has a Some winning rule.
	
	\item \textbf{DestroyGame:} The game has five main objects (Player, Crate, Gem, Target, and Wall). Gem and Target are the winning objects with Gem having Destroy behavior. Crate is a critical rule object with Move and Destroy behaviors. Wall is a solid object. The game has a No winning rule.
\end{itemize}

\subsubsection{Level Layouts}
The second input for the system is the level layouts. Eight different level layouts are used to generate levels. \figref{levelLayouts} shows these layouts. The layouts have different sizes and different internal structure to test the generator with different inputs. The biggest level layout is $8x8$ tiles because as the level size increase, the system takes longer time to generate a good level. The generation time is restricted so less interesting levels are generated. Human statistics proves this point in certain games.

\gfigure{Different level layouts for level generation}{levelLayouts}{width=5.5in}{Images/levelLayouts}

\subsection{Constructive Approach Results}
Hundred level is generated using Constructive Algorithm. These levels are evaluated and the best two levels are selected. This approach is repeated for the five different games and applied over the eight different layouts. The total amount of generated levels are 80 levels. \figref{constructiveExamples} shows examples of some of generated levels for different games. All the generated levels can be played online\footnote{http://www.amidos-games.com}.\\\par

\gfigure{Examples of the generated levels using constructive approach}{constructiveExamples}{width=5.5in}{Images/constructiveExamples}

Out of the 80 levels only 15\% are unplayable by the automated player but having a very high score. By testing these levels by human users only 10\% of these are completely unplayable while the rest 5\% are very challenging levels. \tabref{constructiveScores} shows the score for all the generated levels using the level evaluator and human evaluation.\\\par

There is a huge correlation between the generated scores and the human scores for Sokoban and BlockFaker games than any other games. The reason behind that is the automated player plays these games better than others using 5000 states (refer to \figref{automatedPlayerPerformance}). LavaGame has the most unplayable levels, because these unplayable games have high values in other heuristics compared with its playability score. GemGame and DestroyGame have a very small correlation because they are very hard games so the automated player always prefer the easy levels than hard ones. The reason behind that is the playability metric has huge effect on the final score as Do Nothing Score is always zero which is not the case with LavaGame and BlockFaker.

\begin{landscape}
\begin{table}[!ht]
	\centering
	\begin{tabular}{|p{0.8in}|p{0.8in}|p{0.8in}|p{0.8in}|p{0.8in}|p{0.8in}|p{0.8in}|p{0.8in}|p{0.8in}|p{0.8in}|}
		\hline
		\multicolumn{2}{c}{\textbf{Sokoban}} & \multicolumn{2}{c}{\textbf{LavaGame}} & \multicolumn{2}{c}{\textbf{\textbf{BlockFaker}}} & \multicolumn{2}{c}{\textbf{GemGame}} & \multicolumn{2}{c}{\textbf{DestroyGame}}\\
		\hline
		\textbf{Automated Player} & \textbf{Human Player} & \textbf{Automated Player} & \textbf{Human Player} & \textbf{Automated Player} & \textbf{Human Player} & \textbf{Automated Player} & \textbf{Human Player} & \textbf{Automated Player} & \textbf{Human Player}\\
		\hline
		0.616362 & 0.6 & 0.62840621 & 0 & 0.67445767 & 0.533333333 & 0.770599632 & 0.8 & 0.8232172 & 0.8\\
		\hline
		0.616362 & 0.6 & 0.5876679 & 0.533333333 & 0.6232392 & 0.533333333 & 0.770599 & 0.8 & 0.8232172 & 0.8\\
		\hline
		0.69734 & 0.768421053 & 0.5896824 & 0 & 0.65679029 & 0.533333333 & 0.9577728 & 0.8 & 0.9098534 & 0.8\\
		\hline
		0.694114 & 0.705263158 & 0.5821967 & 0.6 & 0.650513102 & 0.533333333 & 0.95777289 & 0.8 & 0.9095344 & 0.8\\
		\hline
		0.685905 & 0.811111111 & 0.62574735 & 0.466666667 & 0.60716514 & 0.6 & 0.96990456 & 0.6 & 0.914115 & 0.8\\
		\hline
		0.671965 & 0.8125 & 0.61386285 & 0.6 & 0.59945915 & 0.733333333 & 0.95356176 & 0.8 & 0.8975684 & 0.8\\
		\hline
		0.68588 & 0.677777778 & 0.6865308 & 0.6 & 0.743085297 & 0.666666667 & 0.85842389 & 0.6 & 0.9479144 & 0.6\\
		\hline
		0.67818 & 0.711111111 & 0.662806064 & 0.6 & 0.712869944 & 0.733333333 & 0.8531832 & 0.6 & 0.9418275 & 0.6\\
		\hline
		0.676906 & 0.694736842 & 0.7116651 & 0.533333333 & 0.750199824 & 0.866666667 & 0.91531076 & 0.6 & 0.9259153 & 0.6\\
		\hline
		0.6724738 & 0.705263158 & 0.6679423 & 0 & 0.74666923 & 0.8 & 0.8580357 & 0.6 & 0.9144401 & 0.8\\
		\hline
		0.70278 & 0.688888889 & 0.651134564 & 0.666666667 & 0.6585878 & 0 & 0.87784682 & 0.4 & 0.935399 & 0.6\\
		\hline
		0.698841 & 0.866666667 & 0.63963322 & 0 & 0.6572663 & 0.733333333 & 0.8774374 & 0.6 & 0.9293 & 0.6\\
		\hline
		0.659504 & 0.633333333 & 0.77533756 & 0.9 & 0.783839585 & 0.8 & 0.9196736 & 0.8 & 0.947915 & 0.6\\
		\hline
		0.6530206 & 0.705882353 & 0.67766235 & 0.6 & 0.7277456 & 0.8 & 0.8669255 & 0.8 & 0.9369172 & 0.8\\
		\hline
		0.681864 & 0.788888889 & 0.69552052 & 0 & 0.7283029 & 0.733333333 & 0.91133042 & 0.6 & 0.9449218 & 0.6\\
		\hline
		0.671696 & 0.766666667 & 0.667115002 & 0 & 0.7209062 & 0 & 0.89999055 & 0.8 & 0.93881904 & 0.8\\
		\hline
		\textbf{Correlation} & 0.6634913 & \textbf{Correlation} & 0.2545798 & \textbf{Correlation} & 0.2734119 & \textbf{Correlation} & -0.0026173 & \textbf{Correlation} & -0.560329\\
		\hline
	\end{tabular}
	\caption{Automated Scores vs Human Scores for constructive approach}
	\label{Table:constructiveScores}
\end{table}
\end{landscape}

The constructive technique is a very fast technique that can be used for level generation in real time depending on the time of the automated player used. Although the algorithm ensures high level playability, it limits the search space for potential levels producing similar generated levels. The search space is limited due to the following:
\begin{itemize} \itemsep0pt \parskip0pt \parsep0pt
	\item The number of objects are always limited by the size of the free areas in the map. That's why the two generated levels always have same number of objects.
	\item The moving object should be placed at the most free place. That's why the moving objects are always placed at certain places in a certain order.
	\item The second winning object is always at the farthest distance from the first object. That's why the second winning object is always aligned with level borders refer to \figref{constructiveExamples}.
\end{itemize}

Smaller levels are subjected to higher similarity that bigger size levels. Out of the 80 levels only 8.75\% are the same and all of them happens in the first three layouts specially the first level.

\subsection{Genetic Approach Results}
Genetic approach is used to give the system the ability to modify the generated levels to improve the playability and variety of games. Genetic algorithm is used for 20 generations with population of size 50. The crossover rate is around 70\% and mutation rate is around 10\%. The best two chromosomes of each level layout is published.\\\par

The maximum fitness of the GA is presented in \figref{gaWithoutElitism}. Its clear that the best chromosomes disappear after few generations because they have really near fitness. The changes are not so huge but to ensure having best chromosome Elitism is used with probability equals 2\%.

\gfigure{Maximum fitness for GA without Elitism}{gaWithoutElitism}{width=6.0in}{Images/gaWithoutElitism}

The drawback of using elitism that in most of the time the second best chromosome is the same as the best one which creates lots of redundant levels.\\\par

\subsubsection{Random Initialization}
Different initialization methods are tested alone and against each other. The first method is Random Initialization described in \secref{geneticApproach}. \figref{randomMaxFitness} and \figref{randomAverageFitness} shows the evolution of maximum and average fitness respectively. Sokoban is the best evolved game in all of them. That's due to the small amount of objects needed to have a playable level, while GemGame requires pretty tough behavior to be playable that is why it take very long time to evolve. Although DestroyGame have a higher score than Sokoban but it has less quality. The DestroyGame has a zero Do Nothing score, while Sokoban always has a high Do Nothing score (refer to \tabref{constructiveScores}).

\gfigure{Maximum fitness for GA with random initialization}{randomMaxFitness}{width=6.0in}{Images/randomMaxFitness}

\gfigure{Average fitness for GA with random initialization}{randomAverageFitness}{width=6.0in}{Images/randomAverageFitness}

\gfigure{Examples of the generated levels using GA with random initialization}{randomExamples}{width=5.5in}{Images/randomExamples}

\figref{randomExamples} shows a group some of the different levels using GA with random initialization. These levels are subjected to human players to test the playability and quality of each generated level. About 42.5\% of the generated levels are repeated levels and 75\% of them are playable although the automated player reported 26.25\% unplayable levels. \tabref{randomScores} shows the correlation between the automated score and human scores. Most of the unplayable levels are found in the GemGame and DestroyGame, while most playable levels are LavaGame and BlockFaker. The reason behind that is that winning condition for BlockFaker and LavaGame is more easier than GemGame and DestroyGame as they required a certain number of objects to finish it.\\\par

The correlation is the highest at GemGame and DestroyGame because most of the generated levels are unplayable and their best state is the starting state. That is different from unplayable levels in constructive approach where it gets a higher score which indicates the percentage of level playability.

\begin{landscape}
\begin{table}[!ht]
	\centering
	\begin{tabular}{|p{0.8in}|p{0.8in}|p{0.8in}|p{0.8in}|p{0.8in}|p{0.8in}|p{0.8in}|p{0.8in}|p{0.8in}|p{0.8in}|}
		\hline
		\multicolumn{2}{c}{\textbf{Sokoban}} & \multicolumn{2}{c}{\textbf{LavaGame}} & \multicolumn{2}{c}{\textbf{\textbf{BlockFaker}}} & \multicolumn{2}{c}{\textbf{GemGame}} & \multicolumn{2}{c}{\textbf{DestroyGame}}\\
		\hline
		\textbf{Automated Player} & \textbf{Human Player} & \textbf{Automated Player} & \textbf{Human Player} & \textbf{Automated Player} & \textbf{Human Player} & \textbf{Automated Player} & \textbf{Human Player} & \textbf{Automated Player} & \textbf{Human Player}\\
		\hline
		0.472927827 & 0.566666667 & 0.55654906 & 0.9 & 0.56366794 & 0.4 & 0.2895113 & 0 & 0.823277724 & 0.7\\
		\hline
		0.472927827 & 0.566666667 & 0.55654906 & 0.9 & 0.56366794 & 0.4 & 0.2895113 & 0 & 0.823277724 & 0.7\\
		\hline
		0.66605524 & 0.630769231 & 0.4829044 & 0.4 & 0.58819125 & 0.4 & 0.2852678 & 0 & 0.285267846 & 0\\
		\hline
		0.66605524 & 0.630769231 & 0.4829044 & 0.4 & 0.58819125 & 0.4 & 0.2852678 & 0 & 0.285267846 & 0\\
		\hline
		0.620078 & 0.892307692 & 0.58316514 & 0.5 & 0.539436756 & 0.4 & 0.284260212 & 0 & 0.296260212 & 0\\
		\hline
		0.601255799 & 0.892307692 & 0.58316514 & 0.5 & 0.539436756 & 0.4 & 0.284260212 & 0 & 0.296260212 & 0\\
		\hline
		0.57708091 & 0.516666667 & 0.55400489 & 0.4 & 0.50191485 & 0.4 & 0.3350501 & 0 & 0.75341078 & 0.4\\
		\hline
		0.57708091 & 0.516666667 & 0.55400489 & 0.4 & 0.50191485 & 0.4 & 0.3350501 & 0 & 0.75341078 & 0.4\\
		\hline
		0.5720216 & 0 & 0.4216133 & 0.4 & 0.523848613 & 0.4 & 0.346181665 & 0 & 0.87578933 & 0.6\\
		\hline
		0.5720216 & 0 & 0.4216133 & 0.4 & 0.46675503 & 0.4 & 0.33422192 & 0 & 0.87108038 & 0.4\\
		\hline
		0.604114365 & 0.44 & 0.57729139 & 0.6 & 0.50550698 & 0.5 & 0.282968 & 0 & 0.865837091 & 0.6\\
		\hline
		0.604114365 & 0.44 & 0.57729139 & 0.6 & 0.50550698 & 0.5 & 0.282968 & 0 & 0.858546414 & 0.6\\
		\hline
		0.638117335 & 0.633333333 & 0.569474277 & 1 & 0.46041324 & 0.4 & 0.8438745 & 0 & 0.934646319 & 0.6\\
		\hline
		0.638117336 & 0.633333333 & 0.569474277 & 1 & 0.46041324 & 0.4 & 0.8438745 & 0.4 & 0.934646319 & 0.6\\
		\hline
		0.620584613 & 0.771428571 & 0.5843821 & 1 & 0.54673306 & 0.4 & 0.90855179 & 0.8 & 0.84207603 & 0.8\\
		\hline
		0.620584613 & 0.771428571 & 0.5843821 & 1 & 0.53355297 & 0.8 & 0.90855179 & 0.6 & 0.52655178 & 0\\
		\hline
		\textbf{Correlation} & 0.281768023 & \textbf{Correlation} & 0.557156502 & \textbf{Correlation} & -0.0011142 & \textbf{Correlation} & 0.824985501 & \textbf{Correlation} & 0.916474873\\
		\hline
	\end{tabular}
	\caption{Automated Scores vs Human Scores for GA with random initialization}
	\label{Table:randomScores}
\end{table}
\end{landscape}

\subsubsection{Constructive Initialization}
The problem with Random Initialization that it needs very huge amount of generation to converge to a playable challenging levels. Using constructive algorithm to initialize the GA ensures better playability as 90\% of the generated level output is playable levels. GA will ensure the increase of that playability to reach 100\% and it expands the search space finding more diverse results than Constructive Approach only. \figref{gaConstructiveMax} and \figref{gaConstructiveAverage} show the evolution of max fitness and average fitness across the generations. The slow increase in the max fitness score is due to two reasons:
\begin{itemize} \itemsep0pt \parskip0pt \parsep0pt
	\item The high quality of the generated levels by the constructive algorithm.
	\item The limited search space of the constructive algorithm.
\end{itemize}
The second point is the main reason why the average score drops at the beginning. At the beginning, GA expands the search space using the Mutation operators to find different levels. Crossover has a little bit effect as all the levels have same formate of the constructive algorithm. The same reasons above makes the generated levels very similar to the constructive algorithm output. For examples, Sokoban levels have crates set at the most empty spaces, ending targets on the borders, and ...etc. Even the unplayable levels seems like an alternated versions of the constructive algorithm levels.\\\par

\gfigure{Maximum fitness for GA with constructive initialization}{gaConstructiveMax}{width=6.0in}{Images/gaConstructiveMax}

\gfigure{Average fitness for GA with constructive initialization}{gaConstructiveAverage}{width=6.0in}{Images/gaConstructiveAverage}

\figref{constructiveGAExamples} shows some generated levels. 46.25\% of the generated levels are similar due elitism and small search space. All the levels generated are playable. \tabref{constructiveGAScores} shows the automated player score with human scores and calculates the correlation between both of them. Same small correlation happens in the DestroyGame and GemGame due to its difficulty.

\gfigure{Examples of the generated levels using GA with constructive initialization}{constructiveGAExamples}{width=5.5in}{Images/constructiveGAExamples}

\begin{landscape}
\begin{table}[!ht]
	\centering
	\begin{tabular}{|p{0.8in}|p{0.8in}|p{0.8in}|p{0.8in}|p{0.8in}|p{0.8in}|p{0.8in}|p{0.8in}|p{0.8in}|p{0.8in}|}
		\hline
		\multicolumn{2}{c}{\textbf{Sokoban}} & \multicolumn{2}{c}{\textbf{LavaGame}} & \multicolumn{2}{c}{\textbf{\textbf{BlockFaker}}} & \multicolumn{2}{c}{\textbf{GemGame}} & \multicolumn{2}{c}{\textbf{DestroyGame}}\\
		\hline
		\textbf{Automated Player} & \textbf{Human Player} & \textbf{Automated Player} & \textbf{Human Player} & \textbf{Automated Player} & \textbf{Human Player} & \textbf{Automated Player} & \textbf{Human Player} & \textbf{Automated Player} & \textbf{Human Player}\\
		\hline
		0.616362 & 0.6 & 0.6191616 & 0.5 & 0.674457 & 0.6 & 0.8730908 & 0.8 & 0.8769583 & 0.8\\
		\hline
		0.615881 & 0.52 & 0.6191616 & 0.5 & 0.674457 & 0.6 & 0.8730908 & 0.8 & 0.8769583 & 0.8\\
		\hline
		0.71161371 & 0.766666667 & 0.6521912 & 0.6 & 0.640292 & 0.6 & 0.957772 & 0.8 & 0.927642 & 0.6\\
		\hline
		0.71161371 & 0.766666667 & 0.6121912 & 0.6 & 0.640292 & 0.6 & 0.957772 & 0.8 & 0.927642 & 0.6\\
		\hline
		0.6813152 & 0.6 & 0.6261598 & 0.8 & 0.613533 & 0.6 & 0.9541082 & 0.6 & 0.9201257 & 0.8\\
		\hline
		0.6813152 & 0.6 & 0.6260663 & 0.6 & 0.60379 & 0.6 & 0.9541082 & 0.6 & 0.9201257 & 0.8\\
		\hline
		0.692882 & 0.88 & 0.650007 & 0.7 & 0.75497 & 0.8 & 0.92124517 & 0.6 & 0.94662662 & 0.6\\
		\hline
		0.692882 & 0.88 & 0.650007 & 0.7 & 0.75497 & 0.8 & 0.92124517 & 0.6 & 0.94332112 & 0.6\\
		\hline
		0.690599 & 0.6 & 0.765587 & 0.9 & 0.80884 & 0.8 & 0.9003137 & 0.6 & 0.928826 & 0.8\\
		\hline
		0.690599 & 0.6 & 0.765587 & 0.9 & 0.80884 & 0.8 & 0.9003137 & 0.6 & 0.928826 & 0.8\\
		\hline
		0.702111 & 0.666666667 & 0.80019 & 0.7 & 0.827012 & 0.9 & 0.9340367 & 0.8 & 0.9418921 & 0.6\\
		\hline
		0.69881 & 0.666666667 & 0.80019 & 0.7 & 0.827012 & 0.9 & 0.9340367 & 0.8 & 0.9418921 & 0.6\\
		\hline
		0.672561 & 0.766666667 & 0.7755379 & 0.6 & 0.810844 & 0.8 & 0.9438273 & 0.6 & 0.93639401 & 0.8\\
		\hline
		0.672561 & 0.766666667 & 0.7755379 & 0.6 & 0.766883 & 0.8 & 0.9438273 & 0.6 & 0.93639401 & 0.8\\
		\hline
		0.703393 & 0.733333333 & 0.697292 & 0.6 & 0.817545 & 0.8 & 0.923987 & 0.6 & 0.9321509 & 0.8\\
		\hline
		0.703393 & 0.733333333 & 0.697292 & 0.6 & 0.817545 & 0.8 & 0.923987 & 0.6 & 0.9321509 & 0.8\\
		\hline
		\textbf{Correlation} & 0.515628369 & \textbf{Correlation} & 0.42035216 & \textbf{Correlation} & 0.941391762 & \textbf{Correlation} & -0.128466 & \textbf{Correlation} & -0.465830\\
		\hline
	\end{tabular}
	\caption{Automated Scores vs Human Scores for GA with constructive initialization}
	\label{Table:constructiveGAScores}
\end{table}
\end{landscape}

\subsubsection{Mixed Initialization}
The generated games are 100\% playable but most of them still have same structure of The constructive algorithm. To ensure more diversity, Mixed initialization is used. 25\% of population is initialized using constructive algorithm, another 25\% with mutated versions of levels from constructive algorithm, while the rest are mutated versions of level layouts. The mutated levels are the result of subjecting each level to mutation for twenty times.\\\par 

\figref{gaMixedMax} and \figref{gaMixedAverage} shows the evolution of fitness across generations. The average fitness increases along the generation, while max fitness increases very slowly.

\gfigure{Maximum fitness for GA with mixed initialization}{gaMixedMax}{width=6.0in}{Images/gaMixedMax}

\gfigure{Average fitness for GA with mixed initialization}{gaMixedAverage}{width=6.0in}{Images/gaMixedAverage}

\figref{mixedExamples} shows different generated levels using mixed initialization. All levels are playable with 42.5\% repeated levels.

\gfigure{Examples of the generated levels using GA with mixed initialization}{mixedExamples}{width=5.5in}{Images/mixedExamples}

\begin{landscape}
\begin{table}[!ht]
	\centering
	\begin{tabular}{|p{0.8in}|p{0.8in}|p{0.8in}|p{0.8in}|p{0.8in}|p{0.8in}|p{0.8in}|p{0.8in}|p{0.8in}|p{0.8in}|}
		\hline
		\multicolumn{2}{c}{\textbf{Sokoban}} & \multicolumn{2}{c}{\textbf{LavaGame}} & \multicolumn{2}{c}{\textbf{\textbf{BlockFaker}}} & \multicolumn{2}{c}{\textbf{GemGame}} & \multicolumn{2}{c}{\textbf{DestroyGame}}\\
		\hline
		\textbf{Automated Player} & \textbf{Human Player} & \textbf{Automated Player} & \textbf{Human Player} & \textbf{Automated Player} & \textbf{Human Player} & \textbf{Automated Player} & \textbf{Human Player} & \textbf{Automated Player} & \textbf{Human Player}\\
		\hline
		0.654089218 & 0.7 & 0.5989727 & 0.7 & 0.5658553 & 0.6 & 0.82325022 & 0.8 & 0.680589 & 0.4\\
		\hline
		0.654089218 & 0.7 & 0.5989727 & 0.7 & 0.5658553 & 0.6 & 0.82325022 & 0.8 & 0.680589 & 0.4\\
		\hline
		0.64291055 & 0.644444444 & 0.5544389 & 0.8 & 0.6580676 & 0.6 & 0.9621 & 0.8 & 0.9132655 & 0.8\\
		\hline
		0.64291055 & 0.64444444 & 0.5544389 & 0.8 & 0.6580676 & 0.6 & 0.9621 & 0.8 & 0.9132655 & 0.8\\
		\hline
		0.68702434 & 0.533333333 & 0.5526414 & 0.7 & 0.62174446 & 0.6 & 0.9430214 & 0.8 & 0.9396707 & 0.8\\
		\hline
		0.68702434 & 0.533333333 & 0.5526414 & 0.7 & 0.62174446 & 0.6 & 0.9430214 & 0.8 & 0.9390626 & 0.8\\
		\hline
		0.68636698 & 0.555555556 & 0.6992303 & 0.4 & 0.68210815 & 0.7 & 0.8828038 & 0.6 & 0.933516 & 0.6\\
		\hline
		0.6863669 & 0.555555556 & 0.6992303 & 0.4 & 0.68210815 & 0.7 & 0.8824508 & 0.6 & 0.933516 & 0.6\\
		\hline
		0.6894032 & 0.666666667 & 0.718745 & 0.8 & 0.7304408 & 0.8 & 0.924357 & 0.8 & 0.9286588 & 0.4\\
		\hline
		0.6894032 & 0.666666667 & 0.718745 & 0.8 & 0.7304408 & 0.8 & 0.924357 & 0.8 & 0.9286588 & 0.4\\
		\hline
		0.6994754 & 0.688888889 & 0.654869 & 0.7 & 0.7425093 & 0.9 & 0.931827 & 0.8 & 0.933017 & 1\\
		\hline
		0.6994754 & 0.688888889 & 0.654869 & 0.7 & 0.7425093 & 0.9 & 0.931827 & 0.8 & 0.933017 & 1\\
		\hline
		0.67699273 & 0.666666667 & 0.7309346 & 0.6 & 0.65543165 & 0.8 & 0.950839 & 0.6 & 0.9321156 & 0.8\\
		\hline
		0.67699273 & 0.666666667 & 0.7309346 & 0.6 & 0.65543165 & 0.8 & 0.950839 & 0.6 & 0.9321156 & 0.8\\
		\hline
		0.65561072 & 0.755555556 & 0.7776904 & 1 & 0.745198 & 0.8 & 0.93151312 & 0.8 & 0.9451245 & 0.6\\
		\hline
		0.65561072 & 0.755555556 & 0.7586242 & 1 & 0.745198 & 0.8 & 0.93151312 & 0.8 & 0.9451245 & 0.6\\
		\hline
		\textbf{Correlation} & -0.393092 & \textbf{Correlation} & 0.088979799 & \textbf{Correlation} & 0.814104796 & \textbf{Correlation} & 0.026758801 & \textbf{Correlation} &0.511263461\\
		\hline
	\end{tabular}
	\caption{Automated Scores vs Human Scores for GA with mixed initialization}
	\label{Table:mixedGAScores}
\end{table}
\end{landscape}

\subsection{Different Techniques Comparison}

\section{Rule Generation}

\gtable{Example table for demonstration}{tableName}{width=5.0in}{Table3-1}
