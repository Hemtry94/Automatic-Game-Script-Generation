This chapter shows the results of level and the rule generation techniques proposed in \chref{Chapter4}. The generated levels and games are all published on a website \footnote{http://www.amidos-games.com/puzzlescript-pcg/} to collect human feedbacks. The collected feedbacks are compared with the system scores. The following section analyzes the results of the new automated player and compares its results with the original one, followed by the results of the level generation techniques, and the last section shows the results of the rule generation.\\\par

\section{Automated Player}
\secref{automatedPlayer} introduces a new metric to improve the automated player performance. Different weights are given to each one of the three metrics. The distance between winning objects metric is the most important because it indicates the playability of the level. Other metrics contributes with a lower weight equals to 50\%. The distance between the player and the winning objects metric is weighted 50\% compared to the new metric because at least one of the winning objects is a rule object. The new metric measures the distance between rule objects in the left hand side of any rule. This means one of the winning objects already contributes in the new metric. The following equation shows the final weights:
\begin{center}$d_{total} = d_{winning} + 0.5 * (d_{rules} + 0.5 * d_{player})$\end{center}
where $d_{winning}$ is the distance between winning objects metric, $d_{rules}$ is the distance between rule objects (new metric), and $d_{player}$ is the distance between the player and the winning objects.\\\par

The following section will describe the different games and levels used to test the automated players, followed by a comparison between their results.

\subsection{Input Description}\seclbl{gameDescription}
Forty handcrafted levels from five different games were used. The five games are completely different to cover different object behaviors and winning conditions. Levels are also designed with different sizes and ideas to cover different design aspects. \figref{designedLevels} shows different handcrafted levels for these games.

\gfigure{Examples for handcrafted levels from different games}{designedLevels}{width=5.5in}{Images/designedLevels}

The five games are Sokoban, LavaGame, BlockFaker, GemGame, and DestroyGame.
\begin{itemize} \itemsep0pt \parskip0pt \parsep0pt
	\item \textbf{Sokoban:} The goal of the game is to place every single crate over a certain position. Player can push crates to achieve that goal.
	\item \textbf{LavaGame:} The goal of the game is to reach the exit. The path towards the exit is always blocked by a lava. Player should push crates over the lava to clear his way.
	\item \textbf{BlockFaker:} The goal of the game is to reach the exit. The path towards the exit is always blocked by lots of crates. Player should push these crates to align them vertically or horizontally. Every three aligned crates are destroyed which clear the path towards the exit.
	\item \textbf{GemGame:} The goal of the game is to place at least one gem over one of several locations. Player can create gems by pushing crates. Every three aligned crates are replaced with a single gem instead of the middle crate.
	\item \textbf{DestroyGame:} The goal of the game is to clear every single gem. Gems can be destroyed when they are aligned with two other crates vertically or horizontally. Player should push crates to reach that goal.
\end{itemize}

\subsection{Comparing Different Players}
Lim et al.\cite{puzzleScriptGeneration} automated player is compared with the new automated player. Both of them plays all the forty levels and reports the solution length and the number of states explored. \figref{automatedPlayerPerformance} shows the average number of states each player explores in each game to reach the goal.

\gfigure{Comparison between the number of explored states for different automated players}{automatedPlayerPerformance}{width=6.0in}{Images/automatedPlayerPerformance}

The new player outperforms the original player in Sokoban and GemGame, but its almost the similar in the rest of the games. In games where the player is one of the winning objects, the new player just performed slightly better in BlockFaker, while LavaGame is slightly worse. The new players also performed badly in DestroyGame. The main reason behind the bad performance in these games is the presence of the Destroy behavior as the core mechanic of the game. For example, \figref{automatedPlayerProblem} have two crates and two lava, the metric measures the average distance between each crate and all lava. If the lower lava is destroyed first, the metric will represents the long distance between the crate and the far lava which is a little bit longer than before where both distance (short and long) are considered. Based on that, the player will explore more states that will not lead to the destroy of the lower lava.

\gfigure{LavaGame level cause the new player to do worse}{automatedPlayerProblem}{width=2.5in}{Images/automatedPlayerProblem}

Another drawback of the new metric is when levels have lots of unused objects. The new metric will make the player explores the unused objects aiming to reach the goal which increase the total number of explored states. This drawback is not so important as the main goal is to generate levels not to solve them. The presence of unused objects leads to an increase in the level difficulty.\\\par

\figref{automatedPlayerLength} shows the average solution length for each game for the different players. The new player produces  a slightly shorter solutions than the original player. Comparing both \figref{automatedPlayerPerformance} and \figref{automatedPlayerLength} a correlation can be noticed between both of them except for Sokoban. Sokoban does not follow the same pattern due to being abstract as the game has a very small amount of objects and just one rule. This abstraction is the main reason both players can reach the goal in almost the same amount of steps.

\gfigure{Comparison between the average solution length for different automated players}{automatedPlayerLength}{width=6.0in}{Images/automatedPlayerLength}

\tabref{playerCorrelation} shows the correlation between the average solution length and the average explored states between the two players. The values in the table is the difference between the original player and the new player. Positive values indicates the new player is better than the original and vice versa.

\begin{table}[!ht]
	\centering
	\begin{tabular}{|c|c|}
		\hline
		\textbf{The average solution length} & \textbf{The average explored states}\\
		\hline
		-0.875 & 11127.625\\
		\hline
		-1 & -3914\\
		\hline
		5 & 2736.625\\
		\hline
		12.875 & 16763.625\\
		\hline
		-4.75 & -6696.75\\
		\hline
		\textbf{Correlation} & 0.77127\\
		\hline
	\end{tabular}
	\caption{Correlation between the average explored states and the average solution length}
	\label{Table:playerCorrelation}
\end{table}

\section{Level Generation}
This section shows the results of the level generation techniques introduced in \chref{Chapter4}. The new automated player is used with a fixed number of explored states. It is limited to 5000 explored states to ensure fast execution.\\\par

The following section will discuss the required input for the system to work, followed by the results for each technique. The results are evaluated by the system and several human players and a correlation is calculated. The correlation is a bit low for several reasons:
\begin{itemize} \itemsep0pt \parskip0pt \parsep0pt
	\item The automated player evaluates all levels and give them a score corresponding to its percentage of playability which is not the case with human players. 
	\item The very challenging levels are rated high with the system, while low with human players.
	\item There is no single definition for a good level.
	\item Small amount of human players contributes in the evaluation.
\end{itemize}

\subsection{Input Description}
From \chref{Chapter4}, The level generation system needs a game description and level layouts as an input. Five different games were tested with eight different level layouts.

\subsubsection{Game Descriptions}
The five different games are the same games described in \secref{gameDescription} as they cover a wide variety of different object behaviors and winning conditions. The following list explains these behaviors and rules.
\begin{itemize}
	\item \textbf{Sokoban:} The game have four main objects (Player, Crate, Target, and Wall). Crate and Target are winning objects with Crate having a Move behavior. Wall is a solid object. The game has an All winning rule.
	
	\item \textbf{LavaGame:} The game has five main objects (Player, Lava, Crate, Target, and Wall). Player and Target are winning objects. Lava and Crate are normal rule objects with Destroy behavior. Crate also has a Move behavior. Wall is solid object. The game has an All winning rule.
	
	\item \textbf{BlockFaker:} The game has five main objects (Player, Crate, Stopper, Target, and Wall). Player and Target are winning objects. Stopper and Crate are normal rule objects with Crate having Move and Destroy behaviors. Wall is solid object. The game has an All winning rule.
	
	\item \textbf{GemGame:} The game has five main objects (Player, Crate, Gem, Target, and Wall). Gem and Target are winning objects with Gem having a Create behavior. Crate is a critical rule object with Move and Destroy behaviors. Wall is a solid Object. The game has a Some winning rule.
	
	\item \textbf{DestroyGame:} The game has five main objects (Player, Crate, Gem, Target, and Wall). Gem and Target are winning objects with Gem having a Destroy behavior. Crate is a critical rule object with Move and Destroy behaviors. Wall is a solid object. The game has a No winning rule.
\end{itemize}

\subsubsection{Level Layouts}
Eight different level layouts are used to generate levels which are the same like handcrafted games. \figref{levelLayouts} shows these layouts. The layouts have different sizes and different internal structure. The biggest level layout is of size $8x8$ because the generation time increases as the level size increases.

\gfigure{Different level layouts for level generation}{levelLayouts}{width=5.5in}{Images/levelLayouts}

\subsection{Constructive Approach Results}
One hundred level are generated using the constructive algorithm described in \secref{constructiveApproach}. Each level is evaluated and the best two levels are selected. This approach is repeated for the five different games and applied over all the layouts. The total amount of generated levels are 80 levels. \figref{constructiveExamples} shows examples of some of generated levels for different games.\\\par

\gfigure{Examples of the generated levels using constructive approach}{constructiveExamples}{width=5.5in}{Images/constructiveExamples}

Out of the 80 levels only 15\% are reported as unplayable by the system. By testing these levels by human players only 10\% are completely unplayable while the remaining levels are very difficult levels (needs more states to be explored). \tabref{constructiveScores} shows the scores for all the generated levels using the system and human players.\\\par

\tabref{constructiveScores} shows the correlation between the generated scores and the human scores. Sokoban and BlockFaker have the highest correlation value. The reason is the automated player is better at these games using 5000 states (refer to \figref{automatedPlayerPerformance}). LavaGame has the most unplayable levels because of the huge amount of crates compared to the lava as Crate has a higher priority than lava (refer to the game rules). GemGame and DestroyGame have a very small correlation. These games have the most difficult rules so the automated player prefers generating easier levels. These games have a Do Nothing score equals to zero which makes the playability score the biggest influence on the result.

\begin{landscape}
\begin{table}[!ht]
	\centering
	\begin{tabular}{|p{0.8in}|p{0.8in}|p{0.8in}|p{0.8in}|p{0.8in}|p{0.8in}|p{0.8in}|p{0.8in}|p{0.8in}|p{0.8in}|}
		\hline
		\multicolumn{2}{c}{\textbf{Sokoban}} & \multicolumn{2}{c}{\textbf{LavaGame}} & \multicolumn{2}{c}{\textbf{\textbf{BlockFaker}}} & \multicolumn{2}{c}{\textbf{GemGame}} & \multicolumn{2}{c}{\textbf{DestroyGame}}\\
		\hline
		\textbf{Automated Player} & \textbf{Human Player} & \textbf{Automated Player} & \textbf{Human Player} & \textbf{Automated Player} & \textbf{Human Player} & \textbf{Automated Player} & \textbf{Human Player} & \textbf{Automated Player} & \textbf{Human Player}\\
		\hline
		0.616362 & 0.6 & 0.62840621 & 0 & 0.67445767 & 0.533333333 & 0.770599632 & 0.8 & 0.8232172 & 0.8\\
		\hline
		0.616362 & 0.6 & 0.5876679 & 0.533333333 & 0.6232392 & 0.533333333 & 0.770599 & 0.8 & 0.8232172 & 0.8\\
		\hline
		0.69734 & 0.768421053 & 0.5896824 & 0 & 0.65679029 & 0.533333333 & 0.9577728 & 0.8 & 0.9098534 & 0.8\\
		\hline
		0.694114 & 0.705263158 & 0.5821967 & 0.6 & 0.650513102 & 0.533333333 & 0.95777289 & 0.8 & 0.9095344 & 0.8\\
		\hline
		0.685905 & 0.811111111 & 0.62574735 & 0.466666667 & 0.60716514 & 0.6 & 0.96990456 & 0.6 & 0.914115 & 0.8\\
		\hline
		0.671965 & 0.8125 & 0.61386285 & 0.6 & 0.59945915 & 0.733333333 & 0.95356176 & 0.8 & 0.8975684 & 0.8\\
		\hline
		0.68588 & 0.677777778 & 0.6865308 & 0.6 & 0.743085297 & 0.666666667 & 0.85842389 & 0.6 & 0.9479144 & 0.6\\
		\hline
		0.67818 & 0.711111111 & 0.662806064 & 0.6 & 0.712869944 & 0.733333333 & 0.8531832 & 0.6 & 0.9418275 & 0.6\\
		\hline
		0.676906 & 0.694736842 & 0.7116651 & 0.533333333 & 0.750199824 & 0.866666667 & 0.91531076 & 0.6 & 0.9259153 & 0.6\\
		\hline
		0.6724738 & 0.705263158 & 0.6679423 & 0 & 0.74666923 & 0.8 & 0.8580357 & 0.6 & 0.9144401 & 0.8\\
		\hline
		0.70278 & 0.688888889 & 0.651134564 & 0.666666667 & 0.6585878 & 0 & 0.87784682 & 0.4 & 0.935399 & 0.6\\
		\hline
		0.698841 & 0.866666667 & 0.63963322 & 0 & 0.6572663 & 0.733333333 & 0.8774374 & 0.6 & 0.9293 & 0.6\\
		\hline
		0.659504 & 0.633333333 & 0.77533756 & 0.9 & 0.783839585 & 0.8 & 0.9196736 & 0.8 & 0.947915 & 0.6\\
		\hline
		0.6530206 & 0.705882353 & 0.67766235 & 0.6 & 0.7277456 & 0.8 & 0.8669255 & 0.8 & 0.9369172 & 0.8\\
		\hline
		0.681864 & 0.788888889 & 0.69552052 & 0 & 0.7283029 & 0.733333333 & 0.91133042 & 0.6 & 0.9449218 & 0.6\\
		\hline
		0.671696 & 0.766666667 & 0.667115002 & 0 & 0.7209062 & 0 & 0.89999055 & 0.8 & 0.93881904 & 0.8\\
		\hline
		\textbf{Correlation} & 0.6634913 & \textbf{Correlation} & 0.2545798 & \textbf{Correlation} & 0.2734119 & \textbf{Correlation} & -0.0026173 & \textbf{Correlation} & -0.560329\\
		\hline
	\end{tabular}
	\caption{Automated Scores vs Human Scores for constructive approach}
	\label{Table:constructiveScores}
\end{table}
\end{landscape}

The constructive technique is a very fast technique that can be used in real time level generation depending on the time taken by the automated player. Although the algorithm ensures high level playability, it limits the search space for potential levels. The search space is limited due to the restrictions found in the algorithm such as:
\begin{itemize} \itemsep0pt \parskip0pt \parsep0pt
	\item The number of objects are always limited by the size of the free areas in the map. Same map sizes have the same number of objects.
	\item The moving objects are always placed at the most free place. Moving objects are always placed at the same location for each level layout.
	\item The second winning object is always at the farthest distance from the first object. The second winning object is always at the level borders refer to \figref{constructiveExamples}.
\end{itemize}

The smaller levels are subjected to a higher similarity than the bigger ones. Out of the 80 levels only 8.75\% are the same and all of them happens in the first three layouts specially the first one.

\subsection{Genetic Approach Results}
The genetic approach gives the system the ability to modify the generated levels to improve their playability and difficulty. GA is used for 20 generations with a population equal to 50 chromosomes. The crossover rate is around 70\% and the mutation rate is around 10\%. GA is applied on each level layout and the best two chromosomes from each layout are selected.\\\par

Initial results from GA is presented in \figref{gaWithoutElitism}. Its clear that the best chromosomes disappear after few generations so elitism is used with probability equals to 2\%. The drawback of using elitism that the second best chromosome is usually the same as the best one.

\gfigure{Maximum fitness for GA without Elitism}{gaWithoutElitism}{width=6.0in}{Images/gaWithoutElitism}

The following subsections shows the results of the different initialization methods for GA introduced in \secref{geneticApproach}. At the end of each subsection the system scores are compared with human scores, then the results are analyzed.

\subsubsection{Random Initialization}
\figref{randomMaxFitness} and \figref{randomAverageFitness} shows the evolution of the maximum and the average fitness respectively. Sokoban is the best evolved game in all of them because it needs small amount of objects to have a playable level. On the other hand, GemGame levels has very bad score because GemGame has pretty tough restrictions to be playable. The DestroyGame has a zero Do Nothing score, while Sokoban always has a high Do Nothing score (refer to \tabref{constructiveScores}). This difference is the reason that DestroyGame levels has higher fitness score than Sokoban levels but with less difficulty.

\gfigure{Maximum fitness for GA with random initialization}{randomMaxFitness}{width=6.0in}{Images/randomMaxFitness}

\gfigure{Average fitness for GA with random initialization}{randomAverageFitness}{width=6.0in}{Images/randomAverageFitness}

\gfigure{Examples of the generated levels using GA with random initialization}{randomExamples}{width=5.5in}{Images/randomExamples}

\figref{randomExamples} shows different generated levels using GA with random initialization for different games. The generated levels are subjected to human players to test their playability and challenge. About 42.5\% of the generated levels are repeated (due to elitism). 75\% of the generated levels are playable although the automated player reported only 73.75\%. Most of the unplayable levels are found in GemGame and DestroyGame, while most of the playable levels are found in LavaGame and BlockFaker. The reason behind that the winning condition for BlockFaker and LavaGame is more easier to be satisfied than GemGame and DestroyGame.\\\par 

\tabref{randomScores} shows the correlation between the system scores and the human scores. The correlation is the highest at GemGame and DestroyGame because most of the generated levels are unplayable and having a very small system score. It is obvious from the table and the previous figures that more generations are needed to converge to a playable challenging levels. For examples, most of LavaGame and BlockFaker levels are just straight forward towards the goal.

\begin{landscape}
\begin{table}[!ht]
	\centering
	\begin{tabular}{|p{0.8in}|p{0.8in}|p{0.8in}|p{0.8in}|p{0.8in}|p{0.8in}|p{0.8in}|p{0.8in}|p{0.8in}|p{0.8in}|}
		\hline
		\multicolumn{2}{c}{\textbf{Sokoban}} & \multicolumn{2}{c}{\textbf{LavaGame}} & \multicolumn{2}{c}{\textbf{\textbf{BlockFaker}}} & \multicolumn{2}{c}{\textbf{GemGame}} & \multicolumn{2}{c}{\textbf{DestroyGame}}\\
		\hline
		\textbf{Automated Player} & \textbf{Human Player} & \textbf{Automated Player} & \textbf{Human Player} & \textbf{Automated Player} & \textbf{Human Player} & \textbf{Automated Player} & \textbf{Human Player} & \textbf{Automated Player} & \textbf{Human Player}\\
		\hline
		0.472927827 & 0.566666667 & 0.55654906 & 0.9 & 0.56366794 & 0.4 & 0.2895113 & 0 & 0.823277724 & 0.7\\
		\hline
		0.472927827 & 0.566666667 & 0.55654906 & 0.9 & 0.56366794 & 0.4 & 0.2895113 & 0 & 0.823277724 & 0.7\\
		\hline
		0.66605524 & 0.630769231 & 0.4829044 & 0.4 & 0.58819125 & 0.4 & 0.2852678 & 0 & 0.285267846 & 0\\
		\hline
		0.66605524 & 0.630769231 & 0.4829044 & 0.4 & 0.58819125 & 0.4 & 0.2852678 & 0 & 0.285267846 & 0\\
		\hline
		0.620078 & 0.892307692 & 0.58316514 & 0.5 & 0.539436756 & 0.4 & 0.284260212 & 0 & 0.296260212 & 0\\
		\hline
		0.601255799 & 0.892307692 & 0.58316514 & 0.5 & 0.539436756 & 0.4 & 0.284260212 & 0 & 0.296260212 & 0\\
		\hline
		0.57708091 & 0.516666667 & 0.55400489 & 0.4 & 0.50191485 & 0.4 & 0.3350501 & 0 & 0.75341078 & 0.4\\
		\hline
		0.57708091 & 0.516666667 & 0.55400489 & 0.4 & 0.50191485 & 0.4 & 0.3350501 & 0 & 0.75341078 & 0.4\\
		\hline
		0.5720216 & 0 & 0.4216133 & 0.4 & 0.523848613 & 0.4 & 0.346181665 & 0 & 0.87578933 & 0.6\\
		\hline
		0.5720216 & 0 & 0.4216133 & 0.4 & 0.46675503 & 0.4 & 0.33422192 & 0 & 0.87108038 & 0.4\\
		\hline
		0.604114365 & 0.44 & 0.57729139 & 0.6 & 0.50550698 & 0.5 & 0.282968 & 0 & 0.865837091 & 0.6\\
		\hline
		0.604114365 & 0.44 & 0.57729139 & 0.6 & 0.50550698 & 0.5 & 0.282968 & 0 & 0.858546414 & 0.6\\
		\hline
		0.638117335 & 0.633333333 & 0.569474277 & 1 & 0.46041324 & 0.4 & 0.8438745 & 0 & 0.934646319 & 0.6\\
		\hline
		0.638117336 & 0.633333333 & 0.569474277 & 1 & 0.46041324 & 0.4 & 0.8438745 & 0.4 & 0.934646319 & 0.6\\
		\hline
		0.620584613 & 0.771428571 & 0.5843821 & 1 & 0.54673306 & 0.4 & 0.90855179 & 0.8 & 0.84207603 & 0.8\\
		\hline
		0.620584613 & 0.771428571 & 0.5843821 & 1 & 0.53355297 & 0.8 & 0.90855179 & 0.6 & 0.52655178 & 0\\
		\hline
		\textbf{Correlation} & 0.281768023 & \textbf{Correlation} & 0.557156502 & \textbf{Correlation} & -0.0011142 & \textbf{Correlation} & 0.824985501 & \textbf{Correlation} & 0.916474873\\
		\hline
	\end{tabular}
	\caption{Automated Scores vs Human Scores for GA with random initialization}
	\label{Table:randomScores}
\end{table}
\end{landscape}

\subsubsection{Constructive Initialization}
Using constructive algorithm to initialize the GA increase the overall level playability from 90\% to reach 100\% by expanding the search space to find better levels than the constructive approach. \figref{gaConstructiveMax} and \figref{gaConstructiveAverage} show the evolution of the max fitness and the average fitness respectively. The slow increase in the max fitness score is due to:
\begin{itemize} \itemsep0pt \parskip0pt \parsep0pt
	\item The high quality of the generated levels by the constructive algorithm.
	\item The limited search space of the constructive algorithm.
\end{itemize}
The second point is the main reason for the drop in the average score at the beginning. GA uses mutation operator only to expands the search space to find different levels, while crossover has a small effect as all the levels have the same formate of the constructive algorithm. Due to the limited search space, the generated levels are very similar to the constructive approach. For examples, Sokoban level presented in \figref{constructiveGAExamples} has crates at the most empty spaces and targets at the borders like the constructive algorithm.

\gfigure{Maximum fitness for GA with constructive initialization}{gaConstructiveMax}{width=6.0in}{Images/gaConstructiveMax}

\gfigure{Average fitness for GA with constructive initialization}{gaConstructiveAverage}{width=6.0in}{Images/gaConstructiveAverage}

\figref{constructiveGAExamples} shows different generated levels from different games.  All the generated levels are playable. 46.25\% of the generated levels are similar due to the elitism and the small search space. \tabref{constructiveGAScores} shows the system scores with the human scores and calculates the correlation between both of them. The correlation is the same like the constructive approach as most of the generated levels have the same structure of the constructive algorithm levels.

\gfigure{Examples of the generated levels using GA with constructive initialization}{constructiveGAExamples}{width=5.5in}{Images/constructiveGAExamples}

\begin{landscape}
\begin{table}[!ht]
	\centering
	\begin{tabular}{|p{0.8in}|p{0.8in}|p{0.8in}|p{0.8in}|p{0.8in}|p{0.8in}|p{0.8in}|p{0.8in}|p{0.8in}|p{0.8in}|}
		\hline
		\multicolumn{2}{c}{\textbf{Sokoban}} & \multicolumn{2}{c}{\textbf{LavaGame}} & \multicolumn{2}{c}{\textbf{\textbf{BlockFaker}}} & \multicolumn{2}{c}{\textbf{GemGame}} & \multicolumn{2}{c}{\textbf{DestroyGame}}\\
		\hline
		\textbf{Automated Player} & \textbf{Human Player} & \textbf{Automated Player} & \textbf{Human Player} & \textbf{Automated Player} & \textbf{Human Player} & \textbf{Automated Player} & \textbf{Human Player} & \textbf{Automated Player} & \textbf{Human Player}\\
		\hline
		0.616362 & 0.6 & 0.6191616 & 0.5 & 0.674457 & 0.6 & 0.8730908 & 0.8 & 0.8769583 & 0.8\\
		\hline
		0.615881 & 0.52 & 0.6191616 & 0.5 & 0.674457 & 0.6 & 0.8730908 & 0.8 & 0.8769583 & 0.8\\
		\hline
		0.71161371 & 0.766666667 & 0.6521912 & 0.6 & 0.640292 & 0.6 & 0.957772 & 0.8 & 0.927642 & 0.6\\
		\hline
		0.71161371 & 0.766666667 & 0.6121912 & 0.6 & 0.640292 & 0.6 & 0.957772 & 0.8 & 0.927642 & 0.6\\
		\hline
		0.6813152 & 0.6 & 0.6261598 & 0.8 & 0.613533 & 0.6 & 0.9541082 & 0.6 & 0.9201257 & 0.8\\
		\hline
		0.6813152 & 0.6 & 0.6260663 & 0.6 & 0.60379 & 0.6 & 0.9541082 & 0.6 & 0.9201257 & 0.8\\
		\hline
		0.692882 & 0.88 & 0.650007 & 0.7 & 0.75497 & 0.8 & 0.92124517 & 0.6 & 0.94662662 & 0.6\\
		\hline
		0.692882 & 0.88 & 0.650007 & 0.7 & 0.75497 & 0.8 & 0.92124517 & 0.6 & 0.94332112 & 0.6\\
		\hline
		0.690599 & 0.6 & 0.765587 & 0.9 & 0.80884 & 0.8 & 0.9003137 & 0.6 & 0.928826 & 0.8\\
		\hline
		0.690599 & 0.6 & 0.765587 & 0.9 & 0.80884 & 0.8 & 0.9003137 & 0.6 & 0.928826 & 0.8\\
		\hline
		0.702111 & 0.666666667 & 0.80019 & 0.7 & 0.827012 & 0.9 & 0.9340367 & 0.8 & 0.9418921 & 0.6\\
		\hline
		0.69881 & 0.666666667 & 0.80019 & 0.7 & 0.827012 & 0.9 & 0.9340367 & 0.8 & 0.9418921 & 0.6\\
		\hline
		0.672561 & 0.766666667 & 0.7755379 & 0.6 & 0.810844 & 0.8 & 0.9438273 & 0.6 & 0.93639401 & 0.8\\
		\hline
		0.672561 & 0.766666667 & 0.7755379 & 0.6 & 0.766883 & 0.8 & 0.9438273 & 0.6 & 0.93639401 & 0.8\\
		\hline
		0.703393 & 0.733333333 & 0.697292 & 0.6 & 0.817545 & 0.8 & 0.923987 & 0.6 & 0.9321509 & 0.8\\
		\hline
		0.703393 & 0.733333333 & 0.697292 & 0.6 & 0.817545 & 0.8 & 0.923987 & 0.6 & 0.9321509 & 0.8\\
		\hline
		\textbf{Correlation} & 0.515628369 & \textbf{Correlation} & 0.42035216 & \textbf{Correlation} & 0.941391762 & \textbf{Correlation} & -0.128466 & \textbf{Correlation} & -0.465830\\
		\hline
	\end{tabular}
	\caption{Automated Scores vs Human Scores for GA with constructive initialization}
	\label{Table:constructiveGAScores}
\end{table}
\end{landscape}

\subsubsection{Mixed Initialization}
Mixed initialization is used to ensure more diversity by exploring more of the search space. 25\% of population is initialized using the constructive algorithm, another 25\% with mutated versions of the constructive algorithm levels, while the rest are the same like the random initialization (mutated versions of the empty level layout). The mutated levels are constructed by subjecting a level to the mutation operator for twenty times.\\\par 

\figref{gaMixedMax} and \figref{gaMixedAverage} shows the evolution of the max fitness and the average fitness respectively. The average fitness increases steadily along the generations, while the max fitness increases with a very slow rate. Games with high unplayable percentage in the constructive algorithm have the highest increase rate in the maximum fitness such as LavaGame.

\gfigure{Maximum fitness for GA with mixed initialization}{gaMixedMax}{width=6.0in}{Images/gaMixedMax}

\gfigure{Average fitness for GA with mixed initialization}{gaMixedAverage}{width=6.0in}{Images/gaMixedAverage}

\figref{mixedExamples} shows different generated levels using mixed initialization. The generated levels have different structure than the constructive algorithm. For example, LavaGame level in \figref{mixedExamples} have a lot of crates around the goal entity instead of the empty spaces on the right side of the level. All the generated levels are playable with 42.5\% repeated levels. \tabref{mixedGAScores} shows the system scores and human players scores.

\gfigure{Examples of the generated levels using GA with mixed initialization}{mixedExamples}{width=5.5in}{Images/mixedExamples}

\begin{landscape}
\begin{table}[!ht]
	\centering
	\begin{tabular}{|p{0.8in}|p{0.8in}|p{0.8in}|p{0.8in}|p{0.8in}|p{0.8in}|p{0.8in}|p{0.8in}|p{0.8in}|p{0.8in}|}
		\hline
		\multicolumn{2}{c}{\textbf{Sokoban}} & \multicolumn{2}{c}{\textbf{LavaGame}} & \multicolumn{2}{c}{\textbf{\textbf{BlockFaker}}} & \multicolumn{2}{c}{\textbf{GemGame}} & \multicolumn{2}{c}{\textbf{DestroyGame}}\\
		\hline
		\textbf{Automated Player} & \textbf{Human Player} & \textbf{Automated Player} & \textbf{Human Player} & \textbf{Automated Player} & \textbf{Human Player} & \textbf{Automated Player} & \textbf{Human Player} & \textbf{Automated Player} & \textbf{Human Player}\\
		\hline
		0.654089218 & 0.7 & 0.5989727 & 0.7 & 0.5658553 & 0.6 & 0.82325022 & 0.8 & 0.680589 & 0.4\\
		\hline
		0.654089218 & 0.7 & 0.5989727 & 0.7 & 0.5658553 & 0.6 & 0.82325022 & 0.8 & 0.680589 & 0.4\\
		\hline
		0.64291055 & 0.644444444 & 0.5544389 & 0.8 & 0.6580676 & 0.6 & 0.9621 & 0.8 & 0.9132655 & 0.8\\
		\hline
		0.64291055 & 0.64444444 & 0.5544389 & 0.8 & 0.6580676 & 0.6 & 0.9621 & 0.8 & 0.9132655 & 0.8\\
		\hline
		0.68702434 & 0.533333333 & 0.5526414 & 0.7 & 0.62174446 & 0.6 & 0.9430214 & 0.8 & 0.9396707 & 0.8\\
		\hline
		0.68702434 & 0.533333333 & 0.5526414 & 0.7 & 0.62174446 & 0.6 & 0.9430214 & 0.8 & 0.9390626 & 0.8\\
		\hline
		0.68636698 & 0.555555556 & 0.6992303 & 0.4 & 0.68210815 & 0.7 & 0.8828038 & 0.6 & 0.933516 & 0.6\\
		\hline
		0.6863669 & 0.555555556 & 0.6992303 & 0.4 & 0.68210815 & 0.7 & 0.8824508 & 0.6 & 0.933516 & 0.6\\
		\hline
		0.6894032 & 0.666666667 & 0.718745 & 0.8 & 0.7304408 & 0.8 & 0.924357 & 0.8 & 0.9286588 & 0.4\\
		\hline
		0.6894032 & 0.666666667 & 0.718745 & 0.8 & 0.7304408 & 0.8 & 0.924357 & 0.8 & 0.9286588 & 0.4\\
		\hline
		0.6994754 & 0.688888889 & 0.654869 & 0.7 & 0.7425093 & 0.9 & 0.931827 & 0.8 & 0.933017 & 1\\
		\hline
		0.6994754 & 0.688888889 & 0.654869 & 0.7 & 0.7425093 & 0.9 & 0.931827 & 0.8 & 0.933017 & 1\\
		\hline
		0.67699273 & 0.666666667 & 0.7309346 & 0.6 & 0.65543165 & 0.8 & 0.950839 & 0.6 & 0.9321156 & 0.8\\
		\hline
		0.67699273 & 0.666666667 & 0.7309346 & 0.6 & 0.65543165 & 0.8 & 0.950839 & 0.6 & 0.9321156 & 0.8\\
		\hline
		0.65561072 & 0.755555556 & 0.7776904 & 1 & 0.745198 & 0.8 & 0.93151312 & 0.8 & 0.9451245 & 0.6\\
		\hline
		0.65561072 & 0.755555556 & 0.7586242 & 1 & 0.745198 & 0.8 & 0.93151312 & 0.8 & 0.9451245 & 0.6\\
		\hline
		\textbf{Correlation} & -0.393092 & \textbf{Correlation} & 0.088979799 & \textbf{Correlation} & 0.814104796 & \textbf{Correlation} & 0.026758801 & \textbf{Correlation} &0.511263461\\
		\hline
	\end{tabular}
	\caption{Automated Scores vs Human Scores for GA with mixed initialization}
	\label{Table:mixedGAScores}
\end{table}
\end{landscape}

\subsection{Different Techniques Comparison}
All the previous score table are combined and a general correlation is calculated for each game. Sokoban and BlockFaker have a very good correlation which can be seen in \figref{sokobanCorrelation} and \figref{blockFakerCorrelation}.

\gfigure{Correlation between all automated player score and human scores for Sokoban}{sokobanCorrelation}{width=5.5in}{Images/sokobanCorrelation}

\gfigure{Correlation between all automated player score and human scores for BlockFaker}{blockFakerCorrelation}{width=5.5in}{Images/blockFakerCorrelation}

GemGame and DestroyGame have also a high correlation, but since the collected data is very small so \figref{gemGameCorrelation} and \figref{destroyGameCorrelation} looks more flat.

\gfigure{Correlation between all automated player score and human scores for GemGame}{gemGameCorrelation}{width=5.5in}{Images/gemGameCorrelation}

\gfigure{Correlation between all automated player score and human scores for DestroyGame}{destroyGameCorrelation}{width=5.5in}{Images/destroyGameCorrelation}

LavaGame has a very small correlation in \figref{lavaGameCorrelation} because of the bad performance of the automated player and the small amount of the collected data.

\gfigure{Correlation between all automated player score and human scores for LavaGame}{lavaGameCorrelation}{width=5.5in}{Images/lavaGameCorrelation}

The following figures shows a comparison between the max fitness of the all presented techniques. GA with constructive initialization have the highest score in almost all games specially in \figref{blockFakerComparison}, followed by GA with mixed initialization which is almost the same as GA with constructive initialization. Mixed initialization achieves better scores in some levels like in \figref{lavaGameComparison} but with different structure. The worst of them all is the GA with random initialization as it need more generations to find a good playable levels. Its worst scores are in \figref{gemGameComparison} and \figref{destroyGameComparison} due to the difficulty of the rules to reach the winning condition. Constructive algorithm technique is always better than GA with random initialization but rarely outperform the others like in \figref{destroyGameComparison}. The fact that GA uses constructive algorithm as an initial seed to find better levels what makes it outperform the constructive approach. Sokoban scores in \figref{sokobanComparison} are almost the similar with all techniques due to simplicity of the game rules and the small number of objects needed to have a playable level.

\gfigure{Max fitness of all proposed techniques for Sokoban}{sokobanComparison}{width=6.0in}{Images/sokobanComparison}

\gfigure{Max fitness of all proposed techniques for LavaGame}{lavaGameComparison}{width=6.0in}{Images/lavaGameComparison}

\gfigure{Max fitness of all proposed techniques for BlockFaker}{blockFakerComparison}{width=6.0in}{Images/blockFakerComparison}

\gfigure{Max fitness of all proposed techniques for GemGame}{gemGameComparison}{width=6.0in}{Images/gemGameComparison}

\gfigure{Max fitness of all proposed techniques for DestroyGame}{destroyGameComparison}{width=6.0in}{Images/destroyGameComparison}

From the previous figures its clear that Constructive Algorithm always generate a challenging playable levels regardless of the input rules. That point is the main reason behind using the constructive algorithm in Rule Generation besides being a fast generation technique.

\section{Rule Generation}
This section shows the results of the rule generation technique introduced in \chref{Chapter4}. The new automated player is used with a fixed number of explored states. It is limited to 1500 explored states to ensure fast execution.\\\par

GA is used using 50 chromosomes for 50 generations, with crossover rate equals 65\%, mutation rate equals 10\%, and elitism rate equals 2\%. Different experiments are done on the technique. Constructive approach is used to generate 25 levels where the best five levels are selected.\\\par

Different experiments are tested on the system. These experiments are:
\begin{itemize}
	\item \textbf{No Restriction:} The system has no restriction on the levels, the rules, nor the win condition.
	\item \textbf{Fixed Level:} The system uses a fixed level instead of generating one.
	\item \textbf{Fixed Win Condition:} The system fixes the win condition and search for rules.
	\item \textbf{Fixed Rules:} The system has a fixed set of rules and search for win condition. 
\end{itemize}

\figref{ruleMaxFitness} and \figref{ruleAvgFitness} shows the maximum and the average fitness respectively for all the experiments. The graphs show a very slow increase rate due to the vast majority of unplayable games found in the search space. The max fitness and average fitness of \emph{Fixed Rules} experiment are the highest due to the small search space of win conditions.

\gfigure{Maximum fitness for Rule Generation experiments}{ruleMaxFitness}{width=6.0in}{Images/ruleMaxFitness}

\gfigure{Average fitness for Rule Generation experiments}{ruleAvgFitness}{width=6.0in}{Images/ruleAvgFitness}

The following subsections explain the required input data for each experiments, followed by the special input and the results for each one.

\subsection{Input Description}
The inputs for rule generation are the objects and the collision layers. Eight different objects are used over 4 different collision layers. \figref{ruleGenCollisionLayer} shows the different objects distributed on the collision layers. Background is alone in the lowest layer, while Player and Wall are in the highest layer.

\gfigure{Collision layers for Rule Generation objects}{ruleGenCollisionLayer}{width=3.0in}{Images/ruleGenCollisionLayer}

\figref{ruleGenLevelOutline} shows the level outline used in the level generation. The outline is neither big nor small to ensure fast execution with a wide variety of generated levels.

\gfigure{The level outline used for Rule Generation}{ruleGenLevelOutline}{width=2.5in}{Images/ruleGenLevelOutline}

\subsection{No Restriction}
The output games are playable but they are very trivial. The best generated game has the following rules:
\begin{center}
[ action Player ] -> [ $\wedge$ Object4 ]
\end{center}
\begin{center}
Some Object4 on Object1
\end{center}
The rules in this game is related with the win condition. That is not the case with other generated games, rules have no relation with the goal no can be applied. These games have different rules but all have the same winning condition:
\begin{center}
Some Player on Object1
\end{center}
This winning condition is the main reason for making them playable regardless of the rules involved.

\subsection{Fixed Level}
\figref{ruleFixedLevel} shows the selected level for the experiment. This level is one of the hand crafted levels from Sokoban. The main reason for that experiment to make sure that the level generator is not the drawback in the Rule Generation process.\\\par

\gfigure{Fixed level used for Rule Generation}{ruleFixedLevel}{width=2.5in}{Images/ruleFixedLevel}

The output games are winnable but not playable. Most of them have these rules:
\begin{center}
[ Object4 ] -> [ Object1 ]
\end{center}
\begin{center}
[ Player | \ \ \ \ |  Object2 ] -> [  Object3 | $\vee$ Object5 | \ \ \ \ ]
\end{center}
\begin{center}
All Object4 on Player
\end{center}
The first rule is the main reason that the game is winnable. The winning rule also is not correct as both objects on same collision layer but its the reason for making the having the first rule winnable. These bad results can be noticed from the low fitness for this experiment in \figref{ruleMaxFitness} and \figref{ruleAvgFitness}.

\subsection{Fixed Win Condition}
The win condition is fixed to the same one used is Sokoban:
\begin{center}
All Object1 on Object4
\end{center}

The output games are playable which is better from the previous experiment. The best generated rules are:
\begin{center}
[  Object1 | $\vee$ Player ] -> [ \ \ \ \ | $\vee$ Player ]
\end{center}
This rule destroys Object1 when the player moves beside it. Destroying Object1 cause satisfying for the winning condition. Other generated games are winnable games but not playable, as they have same rule without having the Player in the left hand side.

\subsection{Fixed Rules}
This experiment is easier than all the previous experiments as the search space for win condition is much smaller. The rules were fixed to Sokoban rule:
\begin{center}
[ > Player | Object4 ] -> [ > Player | > Object4 ]
\end{center}

The best output win rule was Sokoban which was predicted:
\begin{center}
All Object1 on Object4
\end{center}
Other win rules are all the same like sokoban but with swapping Object1 and Object4 or having Object2 / Object3 instead of Object1. Also one interesting win rule was 
\begin{center}
No Object1 on Object4
\end{center}
which is easier than Sokoban. That is why it has lower fitness than Sokoban.