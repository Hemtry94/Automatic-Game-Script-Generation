Procedural Content Generation has been in industry since the early days of Video Games. It was used to support huge amount of data with a small footprint due to technical limitations. Although technical difficulties become history and storage is no longer a problem, PCG is still one of the hot topics in Video Games Industry and Research. PCG helps us reduce development time and cost, be creative, and understand the process of creating game content. In this work, we propose a system to help in generating levels for Puzzle Script. Levels are generated without any restriction on the rules. Two different approaches are used with a trade off between speed (Constructive approach) and playability (Genetic approach). These two approaches use a level evaluator that scores the generated levels based on their playability and challenge. The generated levels are assessed by human players statistically. The results shows that  the constructive approach is capable of generating playable levels up to 90\%, while genetic approach can reach up to 100\%. The results shows a high correlation between the system scores and the human scores. Also, we extend Lim et al. work in generating rules for Puzzle Script. The constructive approach is used to generate rules without depending on a certain level. The generated rules are playable which proves the possibility of generating playable rules without any restrictions.