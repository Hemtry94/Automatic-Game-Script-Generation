Procedural Content Generation has been in the industry since the early days of Video Games. It was used to support huge amount of data with a small footprint due to technical limitations. Although technical difficulties became history and storage is no longer a problem, PCG is still one of the hot topics in Video Games' research field. PCG helps in reducing development time and cost, becoming more creative, and understanding the process of creating game content. In this work, a system is proposed to help in generating levels for Puzzle Script. Levels are generated without any restriction on the rules. Two different approaches are used with a trade off between speed (Constructive approach) and playability (Genetic approach). These two approaches use a level evaluator that calculate the scores of the generated levels based on their playability and difficulty. The generated levels are assessed by human players statistically, and the results show that the constructive approach is capable of generating playable levels up to 90\%, while genetic approach can reach up to 100\%. The results also show a high correlation between the system scores and the human scores. Lim et al. work is extended in generating rules for Puzzle Script as well. The constructive approach is used to generate rules without depending on a certain level. The results of the new system proves the possibility of generating playable rules without any restrictions.