Video Games were originally created by small groups of people in their spare time. As time passed Video Games evolved to be a huge multi-billion industry where thousands of people work everyday to create new games. Automation doesn't play huge role in creating games as most of the work is done by humans. Hiring humans ensures producing high quality games but it costs a lot of money and time.

\section{Motivation}
During the early days of Video Games, games were created by few people in their spare time. Most of the time was spent in programming the game, while a small portion was dedicated for graphics, sounds, and music because of the technical limitations of the devices at that time. Although these limitation are no more valid, producing a game takes more time than before. Most of that time is spent on creating content for the game (graphics, music, sounds, levels, and ...etc)\cite{budgetAAA}; for example creating graphics for a huge main stream game may take hundreds of artist working for a year or two. That is why the production cost of a huge game reaches millions of dollars\cite{gameCost}.\\\par

That huge production cost caused lots of Game Studios to shut their doors \cite{gameCloses} and hence creating new game ideas became limited. In spite of technological advancement, we could not reduce the cost and time for creating good game contents because creation process heavily depends on the creative aspect of the human designer. Automating this creative process is somehow difficult as there is no concrete criteria for judging creativity which raises the main question: Can technology help to reduce the time and money used in producing games?

\section{Problem Statement}
Creating content for any creative medium, e.g. games, art, music, movies, and ...etc, is really a tough problem and takes a huge amount of time. For example a famous popular game called \emph{Threes} took around 14 month working on the concept itself \cite{threesTime}. Game market is growing fierce as there is always a demand for new games with new contents. That can not be accomplished very fast because creating new innovative games requires long time.\\\par

Games consists of lots of aspects ranging from game graphics to game rules. It is hard to generate automatically all these aspects at the same time, so this work just addresses Levels and Rules. Level Generation has been in industry for decades since early days of games \cite{pcgFirstGame} but it has always used lots of hacks and it has never been generalized. Same applies to Rule Generation except that there is a very small contribution in that field.\\\par

Complete automation of levels and rules for all types of games seems beyond the reach at the present time, so this work focuses on automatic generation of Puzzle Games because they are accessible by all players, they do not require much time to be played, and do not have any boundaries on their ideas. This is our motive to investigate for new ways in Level and Rule Generation for Puzzle Games.

\section{Objectives}
The aim of this work is to try to understand how our brain works in creating creative content, helping game designers and developers to think outside the box, and minimizing their time for searching for new content by providing them with good diverse seeds for levels and rules. It is not intended to create an Artificial Intelligence that can create a whole game from scratch and sell it afterwards because its too early to think that computers can do that on its own and people are not prepared yet to deal with that kind of Intelligence that can replace them at work.

%\section{Publications}

\section{Organization of the thesis}
The remainder of this thesis is organized as follows. \chref{Chapter2} explains the background needed to understand Procedural Content Generation in Video Games, continued by \chref{Chapter3} explaining any previous work done in that area. \chref{Chapter4} explains different methodologies and techniques used in the proposed system, followed by our results from applying these techniques in \chref{Chapter5}. Finally, \chref{Chapter6} presents our conclusion and future work.