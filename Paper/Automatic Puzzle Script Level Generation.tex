\documentclass[letterpaper]{article}
\usepackage{iccc}
\usepackage{times}
\usepackage{helvet}
\usepackage{courier}

\pdfinfo{
/Title (Automatic Puzzle Script Level Generation)
/Subject (Proceedings of ICCC)
/Author (Ahmed Khalifa and Magda Fayek)}
\title{Automatic Puzzle Script Level Generation}
\author{Ahmed Khalifa \and Magda Fayek\\
Computer Engineering Department\\
Cairo University\\
amidos2002@hotmail.com, magdafayek@ieee.org\\
}
\setcounter{secnumdepth}{0}

\begin{document} 
\maketitle
\begin{abstract}
\begin{quote}
In this paper, we present an approach for automatic generating of levels for games done by \emph{Puzzle Script}, a video game description language that helps people to create puzzle games by Stephan Lavelle\cite{puzzleScript}. We have developed an Algorithm called Rule-based Level Generator (RLG) that generate initial levels for any \emph{Puzzle Script} game based on analysis of the rules and their relation. We tune the inputs and the output levels of the RLG using Genetic Algorithm (GA). The results shows that levels generated from the RLG is near local optimum and GA just make slightly improvement on the results to be more solvable towards the used auto-player.
\end{quote}
\end{abstract}

\section{Introduction}
The {\it ICCC--13 Proceedings} will be printed from electronic
manuscripts submitted by the authors. These must be PDF ({\em Portable
Document Format}) files formatted for 8-1/2$''$ $\times$ 11$''$ paper.

\subsection{Length of Papers}
There are two types of papers: Regular papers and Position papers. Each accepted regular paper is allocated 8 pages in the conference proceedings.  Each accepted position paper is allocated 5 pages in the conference proceedings.  Sub-mitted papers longer than the allocated lengths will be re-turned without review.

\subsection{Word Processing Software}
As detailed below, ICCC has prepared and made available a set of
\LaTeX{} macros and a Microsoft Word template for use in formatting
your paper. If you are using some other word processing software, please follow the format instructions given below and ensure that your final paper looks as much like this sample as possible.

\section{Background \& Related Work}
\LaTeX{} and Word templates that implement these instructions
can be retrieved electronically at {\small \tt http://computationalcreativity.net/iccc2013/}.

\subsection{Layout}
Print manuscripts two columns to a page, in the manner in which these
instructions are printed. The exact dimensions for pages are:
\begin{itemize}
\item left and right margins: 0.75$''$
\item column width: 3.375$''$
\item gap between columns: 0.25$''$
\item top margin---first page: 1.375$''$
\item top margin---other pages: 0.75$''$
\item bottom margin: 1.25$''$
\item column height---first page: 6.625$''$
\item column height---other pages: 9$''$
\end{itemize}

\subsection{Format of Electronic Manuscript}
For the production of the electronic manuscript, you must use Adobe's
{\em Portable Document Format} (PDF). A PDF file can be generated, for
instance, on Unix systems using {\tt ps2pdf} or on Windows systems
using Adobe's Distiller. There is also a website with free software
and conversion services: {\tt http://www.ps2pdf.com/}. For reasons of
uniformity, use of Adobe's {\em Times Roman} font is strongly suggested. In
\LaTeX2e{}, this is accomplished by putting
\begin{quote} 
\mbox{\tt $\backslash$usepackage\{times\}}
\end{quote}
in the preamble.
  
Additionally, you must specify the American {\bf
letter} format (corresponding to 8-1/2$''$ $\times$ 11$''$) when
formatting the paper.

\subsection{Title and Author Information}
Center the title on the entire width of the page in a 15-point bold
font. Below it, center the author name(s) in a 12-point bold font, and
then center the address(es) in a 10-point regular font. Credit to a
sponsoring agency can appear on the first page as a footnote.

\subsubsection{Blind Review}
All papers will be reviewed in a single blind manner.  You are at liberty
to include your affiliation and cite your papers in a natural manner, and
you are also at liberty to anonymise the text if you so desire. In this case,
keeping your identity secret is your responsibility.

\subsection{Abstract}
Place the abstract at the beginning of the first column 3$''$ from the
top of the page, unless that does not leave enough room for the title
and author information. Use a slightly smaller width than in the body
of the paper. Head the abstract with ``Abstract'' centered above the
body of the abstract in a 10-point bold font. The body of the abstract
should be 9-point in the same font as the body of the paper.

The abstract should be a concise, one-paragraph summary describing the
general thesis and conclusion of your paper. A reader should be able
to learn the purpose of the paper and the reason for its importance
from the abstract. The abstract should be no more than 200 words long.

\subsection{Text}
The main body of the text immediately follows the abstract. Use
10-point type in {\em Times Roman} font.

Indent when starting a new paragraph, except after major headings.

\subsection{Headings and Sections}
When necessary, headings should be used to separate major sections of
your paper. (These instructions use many headings to demonstrate their
appearance; your paper should have fewer headings.)

\subsubsection{Section Headings}
Print section headings centered, in 12-point bold type in the style shown in
these instructions. Leave a blank space of approximately 10 points
above and 4 points below section headings.  Do not number sections.

\subsubsection{Subsection Headings}
Print subsection headings left justified, in 12-point bold type. Leave a blank space
of approximately 8 points above and 3 points below subsection
headings. Do not number subsections.

\subsubsection{Subsubsection Headings}
Print subsubsection headings inline in 10-point bold type. Leave a blank
space of approximately 6 points above subsubsection headings. Do not
number subsubsections.

\subsubsection{Special Sections}
You may include an unnumbered acknowledgments section, including
acknowledgments of help from colleagues, financial support, and
permission to publish.

Any appendices directly follow the text and look like sections.  In general, appendices should 
be avoided in ICCC manuscripts.

The references section is headed ``References,'' printed in the same
style as a section heading. A sample list of
references is given at the end of these instructions.  Note the various examples for books, proceedings, multiple authors, etc. Use a consistent
format for references, such as that provided by BiB\TeX{}. The reference
list should not include unpublished work.

\subsection{Citations}
Citations within the text should include the author's last name and
the year of publication, for example~\cite{boden92}.  Append
lowercase letters to the year in cases of ambiguity.  Treat multiple
authors as in the following examples:~\cite{lyu04}
(for more than two authors) and
\cite{veale07} (for two authors).  If the author
portion of a citation is obvious, omit it, e.g.,
Woods~\shortcite{Woods81}.  Collapse multiple citations as
follows:~\cite{UCI,Ruch07,OZ}.

\subsubsection{Using \LaTeX{} and BiBTeX to Create Your References}
At the end of your paper, you can include your reference list by using the following commands (which will insert a heading \textbf{References} automatically):

\begin{footnotesize}
\begin{verbatim}
\bibliographystyle{iccc}
\bibliography{bibfile1,bibfile2,...}
\end{verbatim}
\end{footnotesize}

The list of files in the bibliography command should be the names of your BiBTeX source files (that is, the .bib files referenced in your paper).

The iccc.sty file includes a set of definitions for use in formatting references with BiBTeX. These definitions make the bibliography style fairly close to the one specified previously. To use these definitions, you also need the BiBTeX style file iccc.bst available in the author kit on the ICCC web site.

The following commands are available for your use in citing references:
\begin{description}
\item \verb+\+cite: Cites the given reference(s) with a full citation. This appears as ``(Author Year)'' for one reference, or ``(Author Year; Author Year)'' for multiple references.
\item \verb+\+shortcite: Cites the given reference(s) with just the year. This appears as ``(Year)'' for one reference, or ``(Year; Year)'' for multiple references.
\item \verb+\+citeauthor: Cites the given reference(s) with just the author name(s) and no parentheses.
\item \verb+\+citeyear: Cites the given reference(s) with just the date(s) and no parentheses.
\end{description}

{\bf Warning:} The iccc.sty file is incompatible with the hyperref package. If you use it, your references will be garbled. {\it Do not use hyperref!}

\subsection{Footnotes}
Place footnotes at the bottom of the page in 9-point font.  Refer to
them with superscript numbers.\footnote{This is how your footnotes
should appear.} Separate them from the text by a short
line.\footnote{Note the line separating these footnotes from the
text.} Avoid footnotes as much as possible; they interrupt the flow of
the text.

\section{Level Generation}

Place all illustrations (figures, drawings, tables, and photographs)
throughout the paper at the places where they are first discussed (at the bottom
or top of the page), rather than at the end of the paper. If placed at the bottom or top of
a page, illustrations may run across both columns.

Illustrations must be rendered electronically or scanned and placed
directly in your document. In most cases, it is best to render all illustrations
in black and white; however, since the proceedings are produced and distributed electronically, if color
is important for communicating your message, it may be included. Line weights should
be 1/2-point or thicker.

Number illustrations sequentially. Use references of the following
form: Figure 1, Table 2, etc. Place illustration numbers and captions
under illustrations. Leave a margin of 1/4-inch around the area
covered by the illustration and caption.  Use 9-point type for
captions, labels, and other text in illustrations.
\section{Level Evaluation}

\section{Experimental Results}

\section{Conclusion \& Future Work}

\section{Acknowledgments}
The preparation of these instructions and the \LaTeX{} and Word files was 
facilitated by borrowing from similar documents used for AAAI and IJCAI proceedings.


%\appendix{\LaTeX{} and Word Style Files}\label{stylefiles}

%The \LaTeX{} and Word style files are available on the ICCC-13
%website, {\tt http://computationalcreativity.net/iccc2013/}.
%These style files implement the formatting instructions in this
%document.

%The \LaTeX{} files are {\tt iccc.sty} and {\tt iccc.tex}, and
%the Bib\TeX{} files are {\tt iccc.bst} and {\tt iccc.bib}. The
%\LaTeX{} style file is for version 2e of \LaTeX{}, and the Bib\TeX{}
%style file is for version 0.99c of Bib\TeX{} ({\em not} version
%0.98i).

%The Microsoft Word style file consists of a single template file, {\tt
%iccc.dot}. 

%These Microsoft Word and \LaTeX{} files contain the source of the
%present document and may serve as a formatting sample.  


\bibliographystyle{iccc}
\bibliography{iccc}


\end{document}
